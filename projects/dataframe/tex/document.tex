%&pdflatex
\documentclass[prc,twocolumn,twoside,showpacs,superscriptaddress,floatfix]{revtex4-1}
\bibliographystyle{plain}

% preamble:
\usepackage[utf8]{inputenc}
\usepackage{amsmath}    % need for subequations
\usepackage{graphicx}   % need for figures
%\usepackage{verbatim}   % useful for program listings
%\usepackage{color}      % use if color is used in text
%\usepackage{subfigure}  % use for side-by-side figures
%\usepackage{hyperref}   % use for hypertext links, including those to external documents and URLs
%\usepackage{layouts}    % for getting length measurements


\begin{document}

\title{Temperature effect on neutron capture in tin isotopes}

\author{A. C. Berceanu}
\email{andrei.berceanu@eli-np.ro}
\affiliation{ELI-NP, "Horia Hulubei" National Institute for Physics and Nuclear Engineering,
30 Reactorului Street, RO-077125, Bucharest-Magurele, Romania}

\author{Y. F. Niu}
\affiliation{School of Nuclear Science and Technology, Lanzhou University, Lanzhou 730000, China}

\author{Y. Xu}
\affiliation{ELI-NP, "Horia Hulubei" National Institute for Physics and Nuclear Engineering,
30 Reactorului Street, RO-077125, Bucharest-Magurele, Romania}


\date{\today}

\begin{abstract}
The r-process nucleosynthesis is responsible for the creation of about half of
the atomic nuclei heavier than iron, under extreme density and temperature
conditions.
As such, the temperature dependence of neutron capture cross
sections and rates is important for determining the reaction dynamics.
As shown
by the sensitivity study in~\cite{Mumpower2016}, neutron capture dynamics
on Sn isotopes is very important for r-process study.
 So in this work, we
study the effect of finite temperature on the neutron-capture cross sections and
rates of even-even tin isotopes, with neutron numbers between 76 and 96.
 We
compute the E1 dipole strengths, for both zero and finite temperature, using
relativistic Hartree-Bogoliubov (RHB) + quasiparticle random phase approximation
(QRPA) and finite-temperature relativistic mean-field (FTRMF) +
finite-temperature random-phase approximation (FTRPA), respectively.
We use the
TALYS code for computing the corresponding cross sections, replacing its default
E1 dipole data with ours.
 We find out the main effect of temperature is to
increase the low-lying E1 strength, and as a result, the neutron capture cross
sections are increased several times up to the temperature of 2 MeV\@.
\end{abstract}

\pacs{
21.60.Jz, % Nuclear Density Functional Theory and extensions (includes Hartree-Fock and random-phase approximations)
23.40.Hc, % Relation with nuclear matrix elements and nuclear structure (under item 23.40.-s beta decay)
24.30.Cz, % Giant resonances
25.40.Kv  % Charge-exchange reactions
}

\maketitle


\section{Introduction}\label{sec:intro}

The origin of heavy elements from iron to uranium has been one of the
long-standing open questions in physics~\cite{discover}.
It has been known that
the rapid neutron capture process, or r-process, produces about half of these
heavy elements~\cite{Placeholder:01}.
However, the identification of the sites
for r-process is still under investigation.
Recently, the observation of
GW170817 neutron star mergers provides the evidence that neutron star mergers
can be one of the possible sites~\cite{Placeholder:02}.
In principle, the
different astrophysical sites should produce unique abundance pattern
signatures due to distinct environmental conditions like temperature, density
and neutron densities, which helps us identify the main r-process site
directly\cite{Placeholder:03}.
 For such a purpose, accurate nuclear physics
inputs for r-process network calculation are crucial, including nuclear mass,
$\beta$-decay half-lives, neutron capture rates, and so on~\cite{Placeholder:04}.
However, r-process involves a large amount of
neutron-rich nuclei far from stability line, which are difficult to produce and
measure from experiment.
With the development of radioactive ion beam
facilities, the measurements of nuclear mass and $\beta$-decay half-lives are
approaching towards r-process path~\cite{Placeholder:05}, while the measurement
of neutron-capture cross section is still a big challenge~\cite{Placeholder:06}.
Therefore, the theoretical predictions of
neutron-capture cross sections and rates are necessary for r-process study.
For
the study of neutron-capture process, the statistical Hauser-Feshbach~\cite{Placeholder:07} model is usually adopted, which requires the electric
dipole transition strength function as an important input for this model.

Although the dipole strength function can be conventionally described by a
phenomenological Lorentzian model or its extensions with an energy and
temperature dependent width\cite{Capote2009, Nucl_Data_Sheets_110_3107}, where
the parameters are mainly adjusted to data of stable nuclei, the reliable
extrapolation to neutron rich nuclei requires microscopic models.
It is
difficult for large-scale shell model to describe the E1 transitions since E1
transition connects different major shells~\cite{Placeholder:08}, so a
practical choice is the quasiparticle random phase approximation (QRPA) model
based on density functional theories.
Large-scale QRPA calculations of E1
strength function have been performed based on zero-range Skyrme density
functional, and the dipole strengths are used to estimate the radiative neutron
capture cross section for all nuclei of relevance in astrophysics applications~\cite{Goriely2002, Goriely2004}.
Recently, based on the finite-range Gogny
force, the axially symmetric deformed QRPA approach was applied to the
large-scale calculation of E1 $\gamma$-ray strength function~\cite{Martini2016}, and the resulted $\gamma$-ray strength functions, improved
at low $\gamma$-ray energies by combining shell model results, are used to
study the radioactive neutron and proton capture cross sections~\cite{Goriely2018}.

Compared to the non-relativistic density functional, the covariant density
functional theory (CDFT) has its advantage to produce the spin-orbit splitting
naturally, and has been applied very successfully to the description of a
variety of nuclear structure phenomena~\cite{Placeholder:09}.
Based on the
relativistic density functional, the self-consistent QRPA approach has been
developed and applied to the study of the dipole strength function of some
typical nuclei, including both the giant resonance part and low-lying strength~\cite{Paar2003}.
In neutron rich nuclei, the presence of pygmy dipole resonance
(PDR)~\cite{Paar2007} makes the conventional Lorentzian shape failing to
describe the low-lying part of the dipole strength.
Based on relativistic QRPA
(RQRPA) approach, the properties of PDR have been investigated in detail~\cite{Niu, Vretenar}.
Furthermore, the influence of the enhancement in
low-lying strength on neutron-capture cross sections and rates has been
investigated based on the relativistic quasiparticle time blocking
approximation (QTBA), where the correlations beyond QRPA approach is included,
and it is concluded that the neutron-capture rates are sensitive to the fine
structure of the low-lying dipole strength~\cite{Litvinova_2009}.

Neutron capture relevant for astrophysics processes, like r-process, happens in
stellar environment with high temperature.
However, the temperature effects are
usually not included in the microscopic calculation of dipole strength function
but included as a parameter in the Lorentzian function in a phenominological
way for the calculation of neutron capture rates~\cite{GorielyNPA2002,
GorielyPLB1998, Goriely2019}.
In this case, the fine structure modified by
temperature effects or novel structure induced by temperature is missing in the
calculation of neutron capture rates, which may have an important impact on the
neutron capture cross sections and rates.
Actually, the self-consistent
finite-temperature relativistic random-phase approximation (FTRRPA) model has
been developed, and it is indeed shown that low-lying dipole strengths are
modified by temperature effects, including the concentration of new low-lying
dipole strength for $^{60,62}$Ni, and the modification of PDR for $^{68}$Ni and
$^{132}$Sn~\cite{Niu_2009}.
Later, based on Woods-Saxon mean field, the thermal
continuum QRPA (TCQRPA) model explains the low-energy enhancement of the dipole
strength functions with the inclusion of temperature effect~\cite{Litvinova2013}.
More recently, the self-consistent finite temperature
QRPA based on Skyrme density functional was developed~\cite{Yuksel2017}, and
new low energy dipole excitations are discovered as well~\cite{Yuksel2017,
Yuksel2019}, similar as Ref~\cite{Niu_2009, Litvinova2013}.

With these discoveries, it will be interesting to see what are the consequences
from these modifications of low-lying dipole strength with the microscopic
inclusion of temperature effects on the neutron capture cross sections and
rates.
Therefore, in this work, taking Sn isotopes as examples, we will
calculate the electric dipole strength at finite temperature using
self-consistent FTRRPA model, and correspondingly we will investigate the
influences on neutron capture cross sections and rates using these dipole
strength functions.
The dipole strengths at zero temperature are calculated by
RQRPA model with the inclusion of pairing correlations.
Within the mean field
approach, it exists a critical temperature above which the pairing correlations
vanish~\cite{Niu2013, Yuksel}, although this sharp phase transition is washed
out by including thermal fluctuations beyond mean field~\cite{Gambarcuta}.
This
tells us at relatively high temperatures, the pairing correlations are not so
important any more, so we will use the FTRRPA model for the calculation of
dipole strength functions at finite temperatures.
The neutron capture cross
sections and rates will be calculated by TALYS code.


\section{Formalism}\label{sec:formalism}

The FTRRPA model is formulated based on the single-particle states from the
relativistic mean-field (RMF) model at finite temperature (FTRMF) in a
self-consistent way, which means that the same relativistic density functional
is used for the mean field in FTRMF and for the residual two-body interaction
in FTRRPA. The density-dependent meson-nucleon couplings are used in the
relativistic Lagrangian, and in our calculations we will use the parameter set
DD-ME2~\cite{Lala}.
The FTRRPA equation reads~\cite{Niu_2009}

\begin{equation}
   \left( \begin{array}{cc} A & B \\ -B^* & -A^* \end{array} \right)
   \left( \begin{array}{c} X \\ Y \end{array} \right)
   = \hbar\omega \left( \begin{array}{c} X \\ Y \end{array} \right) \;,
\end{equation}
where
\begin{multline}
   A = \left( \begin{array}{cc} (\epsilon_m - \epsilon_i)
   \delta_{ii'} \delta_{mm'} &  \\
   & (\epsilon_\alpha - \epsilon_i) \delta_{\alpha \alpha'}
   \delta_{ii'} \end{array} \right) + \\
   + \left( \begin{array}{cc} (n_{i'} - n_{m'})V_{mi'im'} & n_{i'} V_{mi'i\alpha'} \\
   (n_{i'} - n_{m'})V_{\alpha i' i m'}  &n_{i'} V_{\alpha i' i \alpha'} \end{array}
   \right) \;,
  \end{multline}
  and
\begin{equation}
   B =\left( \begin{array}{cc} (n_{i'} - n_{m'})V_{mm'ii'} & n_{i'}V_{m\alpha'ii'} \\
    (n_{i'} - n_{m'})V_{\alpha m' i i'}  & n_{i'}  V_{\alpha \alpha' i i' } \end{array}
    \right)
  \end{equation}

$V$ is the residual two-body interaction, derived from the relativistic
Lagrangian with  density-dependent meson-nucleon couplings~\cite{Niksic}.
The
thermal occupation factors $n_k$ denotes a Fermi-Dirac distribution for states
in Fermi sea with index $m,i$, and $0$ for states in Dirac sea with index
$\alpha$.
Due to temperature effects, the configuration space includes not only
particle-hole (ph) pairs, but also particle-particle (pp) and hole-hole (hh)
pairs.
The transition strength for a multipole operator $Q_J$ is

\begin{equation}
   B(J,E_\nu) = \left | \sum_{mi} (X^{\nu,J}_{mi} +
   (-1)^J Y^{\nu,J}_{mi} ) \langle m || Q_J || i\rangle (n_i
   -n_m) \right |^2
\end{equation}

The discrete spectra are averaged with a Lorentzian distribution of an width
$\Gamma = 1 $ MeV in the present calculations.



%-------------------------------------------------------------------------------

\section{Results and Discussions}\label{sec:results}

\begin{figure}[htp]
\centering
\includegraphics[width=\linewidth]{figures/colormesh}
\caption{\label{fig:e1}Computed E1 dipole strengths (logarithmic scale).}
\end{figure}

\begin{figure}[htp]
\centering
\includegraphics[width=\linewidth]{figures/strength_cross_section}
\caption{\label{fig:e1_xsec}E1 strength and neutron capture cross section.}
\end{figure}

\begin{figure}[htp]
\centering
\includegraphics[width=\linewidth]{figures/capture_rate_vs_N}
\caption{\label{fig:neutron_capture}Neutron capture rate.}
\end{figure}

\section{Summary and Perspectives}\label{sec:summary}

{\center{\textbf{ACKNOWLEDGMENTS} }}

This work was supported by Extreme Light Infrastructure-Nuclear
Physics Phase II, a project cofinanced by the Romanian Government
and the European Union through the European Regional Development
Fund and the Competitiveness Operational Program (1/07.07.2016,
COP, ID 1334).

This work was partly supported by the National Natural Science Foundation of China
under Grants No. 11305161 and by JSPS KAKENHI Grant Numbers JP16K05367.
Funding from the European Unions Horizon 2020 research and innovation
programme under grant agreement No. 654002 is also
acknowledged.

\clearpage
\bibliography{document}

\end{document}
